\documentclass[12pt, a4paper, hidelinks]{article}

\usepackage[icelandic]{babel}
\usepackage[T1]{fontenc}
\usepackage[utf8]{inputenc}


\usepackage{amsmath, amssymb, amsfonts}
\usepackage{mathtools}

\usepackage{minted}
\renewcommand{\listingscaption}{Forrit}

\usepackage{url}
\usepackage{hyperref}
\usepackage[hang, flushmargin]{footmisc}
\usepackage[labelfont=sc]{caption}


\usepackage{xcolor}
\usepackage{tabularx}
\usepackage{float}
\usepackage{graphicx}
\usepackage{booktabs}
\usepackage{enumerate}
\usepackage{multirow}
\usepackage{tikz}

\usepackage{times, mathptmx}
\usepackage[scaled=0.85]{beramono}

\usepackage{fancyhdr}
\pagestyle{fancy}
\fancyhf{}
\fancyhead[L]{Kári Hlynsson}
\fancyhead[C]{TÖL203G HEIMADÆMI 5}
\fancyhead[R]{\today}
\fancyfoot[C]{\thepage}

\newcommand{\doctitle}{\uppercase{Heimadæmi 5}}
\newcommand{\coursename}{Tölvunarfræði 2}
\newcommand{\coursenum}{TÖL203G}

% ——— Mengjatákn
\newcommand{\N}{\mathbb{N}}
\newcommand{\Z}{\mathbb{Z}}
\newcommand{\Q}{\mathbb{Q}}
\newcommand{\R}{\mathbb{R}}
\newcommand{\C}{\mathbb{C}}

% ——— Vigrar
\renewcommand{\u}{\mathbf{u}}
\renewcommand{\v}{\mathbf{v}}
\renewcommand{\b}{\mathbf{b}}
\newcommand{\w}{\mathbf{w}}
\newcommand{\p}{\mathbf{p}}
\newcommand{\x}{\mathbf{x}}
\newcommand{\y}{\mathbf{y}}
\newcommand{\z}{\mathbf{z}}

\title{}

\begin{document}
\thispagestyle{plain}
\centerline{\bfseries\Large\doctitle}
\medskip
\centerline{\large\coursenum\ \coursename}
\bigskip

\centerline{\large Kári Hlynsson\footnote{Slóð á Github kóða: \url{https://github.com/lvthnn/TOL203G/tree/master/HD5}}}
\bigskip
\centerline{Háskóli Íslands}
\medskip
\centerline{\today}


\section*{Verkefni 1}
Skoðið Java kóðann fyrir Mergesort (\textsc{Algorithm 2.4} í bókinni, glæra 8 í fyrirlestri 9).
Segjum að við köllum aðeins á \texttt{merge}-fallið (neðsta línan í \texttt{sort}-fallinu) ef
\texttt{a[mid+1]} er minna en \texttt{a[mid]}.
\begin{enumerate}[(a)]
    \item Hvers vegna er í lagi að gera þetta?
    \item Á hvernig inntaki myndum við græða mest á að bæta þessu inn?
\end{enumerate}

\subsection*{Lausn}
\subsubsection*{Hluti (a)}
Hlutverk \texttt{merge} fallsins er að raða hlutfylkjum sem hefur þegar verið raðað endurkvæmt.
Segjum sem svo að $\texttt{a[0]} < \cdots < \texttt{a[mid]}$ og $\texttt{a[mid+1]} < \cdots < \texttt{a[n]}$.
Ef $\texttt{a[mid]} < \texttt{a[mid+1]}$.

\subsubsection*{Hluti (b)}
Við græðum mest á þessari útfærslu ef sérhvert stak í fylkinu er í réttu hlutfylki. Til að sjá þetta betur
skulum við taka dæmi. Látum \texttt{a[] = \{0,4,3,2,1, 8,9,6,5,7\}}. Þegar við köllum á \texttt{sort} byrjum
við á því að sortera vinstri og hægri hlutann og fáum hlutfylkin \texttt{a[0..4] = \{0,1,2,3,4\}} og \texttt{a[5..9] = \{5,6,7,8,9\}}.
Nú er \texttt{a[mid] < a[mid+1]} og við köllum ekki á \texttt{merge} en við sjáum jafnframt að fylkið er raðað svo það er óþarft.

\newpage
\section*{Verkefni 2}
Leysið dæmi 2.3.18 á bls.\@ 305 í kennslubókinni. Kallið ykkur útgáfu \texttt{QuickX}
og berið hana saman við upphaflegu útgáfuna í bókinni (\texttt{Quick.java}) með forritinu 
\texttt{SortCompare.java}. Þið getið hent út úr \texttt{SortCompare} notkun á öðrum röðunaraðferðum.
Skilið breyttu útgáfunni af \texttt{partition}-fallinu og niðurstöðu úr samanburði á \texttt{Quick} og
\texttt{QuickX} í \texttt{SortCompare}. Til að fá raunhæfan samanburð notið \texttt{n = 1000000} og \texttt{trials = 10}.
Þá ættuð þið líka að breyta útprentunarskipuninni í \texttt{SortCompare}, þannig að þið fáið fleiri aukastafi
í hlutfallinu milli tímanna. Það ætti ekki að vera mjög mikill munur á þessum tveimur útgáfum. Hvers vegna?

\subsection*{Lausn}
Útfærslan er til sýnis fyrir neðan í \textsc{Forriti} \ref{forrit:quickx-partition}. \texttt{median} er fall sem var útfært til
að skila miðgildi þriggja staka. Úr því að \texttt{a[]} er stokkað í \texttt{public} útfærslunni af \texttt{sort} notum við fyrsta
stakið í hlutfylkinu, miðjustak og síðasta stak.
\begin{listing}[H]
    \centering
    \inputminted[linenos, bgcolor=lightgray!12, fontsize=\small, firstline=18, lastline=30]{java}{../src/V2/QuickX.java}
    \caption{Útfærslan á \texttt{partition} í \texttt{QuickX} klasanum}
    \label{forrit:quickx-partition}
\end{listing}
\noindent
Munur í tímaflækju á \texttt{Quick} og \texttt{QuickX} er nánast enginn. 10 mælingum var safnað úr \texttt{SortCompare} á hlutfallinu $t_Q/t_X$ þar sem
$t_Q$ er keyrslutími \texttt{Quick} og $t_X$ er keyrslutími \texttt{QuickX}. Meðaltal var $1.0036$ (staðalfrávik $0.066$) svo keyrslutími er nokkurn veginn
sá sami. 

Munurinn er lítill því \texttt{a[]} hefur þegar verið slembistokkað, svo það er alveg tilviljanakennt hvort við drögum úrtak
úr fylkinu sem inniheldur miðjaða stærð, en hraði reikniritsins eykst auðvitað ef skiptistök eru miðjuð.

\newpage

\section*{Verkefni 3}
Fylkinu \texttt{[2,3,1,4,5,7,6,8]} hefur verið skipt upp (\emph{partitioned}) um vendistak (og það sett á réttan stað), en það er ekki gefið upp
hvert vendistakið var. Hver af stökunum gætu hafa verið vendistakið? Teljið upp öll möguleg stök og rökstyðjið að þau séu möguleg og hin séu það ekki.

\end{document}