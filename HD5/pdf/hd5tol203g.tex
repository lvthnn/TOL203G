\documentclass[12pt, a4paper, hidelinks]{article}

\usepackage[icelandic]{babel}
\usepackage[T1]{fontenc}
\usepackage[utf8]{inputenc}


\usepackage{amsmath, amssymb, amsfonts}
\usepackage{mathtools}

\usepackage{minted}
\usemintedstyle{manni}
\renewcommand{\listingscaption}{Forrit}

\usepackage{url}
\usepackage{hyperref}
\usepackage[hang, flushmargin]{footmisc}
\usepackage[labelfont=sc]{caption}


\usepackage[svgnames]{xcolor}
\usepackage{tabularx}
\usepackage{float}
\usepackage{graphicx}
\usepackage{booktabs}
\usepackage{enumerate}
\usepackage{multirow}
\usepackage{tikz}
\usepackage{tikz-qtree}
\usepackage{pifont}
\usepackage{multicol}
\usepackage{tcolorbox}

\newcommand{\cmark}{\text{\color{Green}\ding{51}}}
\newcommand{\xmark}{\text{\color{Red}\ding{55}}}

\usepackage{times, mathptmx}
\usepackage[scaled=0.85]{beramono}

\usepackage{fancyhdr}
\pagestyle{fancy}
\fancyhf{}
\fancyhead[L]{Kári Hlynsson}
\fancyhead[C]{TÖL203G HEIMADÆMI 5}
\fancyhead[R]{\today}
\fancyfoot[C]{\thepage}

\newcommand{\doctitle}{\uppercase{Heimadæmi 5}}
\newcommand{\coursename}{Tölvunarfræði 2}
\newcommand{\coursenum}{TÖL203G}

% ——— Mengjatákn
\newcommand{\N}{\mathbb{N}}
\newcommand{\Z}{\mathbb{Z}}
\newcommand{\Q}{\mathbb{Q}}
\newcommand{\R}{\mathbb{R}}
\newcommand{\C}{\mathbb{C}}

% ——— Vigrar
\renewcommand{\u}{\mathbf{u}}
\renewcommand{\v}{\mathbf{v}}
\renewcommand{\b}{\mathbf{b}}
\newcommand{\w}{\mathbf{w}}
\newcommand{\p}{\mathbf{p}}
\newcommand{\x}{\mathbf{x}}
\newcommand{\y}{\mathbf{y}}
\newcommand{\z}{\mathbf{z}}

\title{}

\begin{document}
\thispagestyle{plain}
\centerline{\bfseries\Large\doctitle}
\medskip
\centerline{\large\coursenum\ \coursename}
\bigskip

\centerline{\large Kári Hlynsson\footnote{Slóð á Github kóða: \url{https://github.com/lvthnn/TOL203G/tree/master/HD5}}}
\bigskip
\centerline{Háskóli Íslands}
\medskip
\centerline{\today}


\section*{Verkefni 1}
Skoðið Java kóðann fyrir Mergesort (\textsc{Algorithm 2.4} í bókinni, glæra 8 í fyrirlestri 9).
Segjum að við köllum aðeins á \texttt{merge}-fallið (neðsta línan í \texttt{sort}-fallinu) ef
\texttt{a[mid+1]} er minna en \texttt{a[mid]}.
\begin{enumerate}[(a)]
    \item Hvers vegna er í lagi að gera þetta?
    \item Á hvernig inntaki myndum við græða mest á að bæta þessu inn?
\end{enumerate}

\subsection*{Lausn}
\subsubsection*{Hluti (a)}
Hlutverk \texttt{merge} fallsins er að raða hlutfylkjum sem hefur þegar verið raðað endurkvæmt.
Segjum sem svo að $\texttt{a[0]} < \cdots < \texttt{a[mid]}$ og $\texttt{a[mid+1]} < \cdots < \texttt{a[n]}$.
Ef $\texttt{a[mid]} < \texttt{a[mid+1]}$.

\subsubsection*{Hluti (b)}
Við græðum mest á þessari útfærslu ef sérhvert stak í fylkinu er í réttu hlutfylki. Til að sjá þetta betur
skulum við taka dæmi. Látum \texttt{a[] = \{0,4,3,2,1, 8,9,6,5,7\}}. Þegar við köllum á \texttt{sort} byrjum
við á því að sortera vinstri og hægri hlutann og fáum hlutfylkin \texttt{a[0..4] = \{0,1,2,3,4\}} og \texttt{a[5..9] = \{5,6,7,8,9\}}.
Nú er \texttt{a[mid] < a[mid+1]} og við köllum ekki á \texttt{merge} en við sjáum jafnframt að fylkið er raðað svo það er óþarft.

\newpage
\section*{Verkefni 2}
Leysið dæmi 2.3.18 á bls.\@ 305 í kennslubókinni. Kallið ykkur útgáfu \texttt{QuickX}
og berið hana saman við upphaflegu útgáfuna í bókinni (\texttt{Quick.java}) með forritinu 
\texttt{SortCompare.java}. Þið getið hent út úr \texttt{SortCompare} notkun á öðrum röðunaraðferðum.
Skilið breyttu útgáfunni af \texttt{partition}-fallinu og niðurstöðu úr samanburði á \texttt{Quick} og
\texttt{QuickX} í \texttt{SortCompare}. Til að fá raunhæfan samanburð notið \texttt{n = 1000000} og \texttt{trials = 10}.
Þá ættuð þið líka að breyta útprentunarskipuninni í \texttt{SortCompare}, þannig að þið fáið fleiri aukastafi
í hlutfallinu milli tímanna. Það ætti ekki að vera mjög mikill munur á þessum tveimur útgáfum. Hvers vegna?

\subsection*{Lausn}
Útfærslan er til sýnis fyrir neðan í \textsc{Forriti} \ref{forrit:quickx-partition}. \texttt{median} er fall sem var útfært til
að skila miðgildi þriggja staka. Úr því að \texttt{a[]} er stokkað í \texttt{public} útfærslunni af \texttt{sort} notum við fyrsta
stakið í hlutfylkinu, miðjustak og síðasta stak.
\begin{listing}[H]
    \centering
    \inputminted[linenos, fontsize=\small, firstline=18, lastline=34, frame=single]{java}{../src/V2/QuickX.java}
    \caption{Útfærslan á \texttt{partition} í \texttt{QuickX} klasanum}
    \label{forrit:quickx-partition}
\end{listing}
\noindent
Munur í tímaflækju á \texttt{Quick} og \texttt{QuickX} er nánast enginn. Meðaltal 10 mælinga úr \texttt{SortCompare} var $1.0036$ (staðalfrávik $0.0659$) svo keyrslutími er nokkurn veginn
sá sami.  Munurinn er lítill því \texttt{a[]} hefur þegar verið slembistokkað og við fáum því ekki nauðsynlega gott (miðlægt) vendistak.

\newpage

\section*{Verkefni 3}
Fylkinu \texttt{[2,3,1,4,5,7,6,8]} hefur verið skipt upp (\emph{partitioned}) um vendistak (og það sett á réttan stað), en það er ekki gefið upp
hvert vendistakið var. Hver af stökunum gætu hafa verið vendistakið? Teljið upp öll möguleg stök og rökstyðjið að þau séu möguleg og hin séu það ekki.

\subsection*{Lausn}
Við getum skoðað hvort stak sé hugsanlegt vendistak út frá eftirfarandi: ef hlutfylkið vinstra megin við stakið inniheldur bara gildi minni en eða jöfn skiptistakinu,
og hægra hlutfylkið bara gildi stærri en eða jöfn því, þá er það gilt. Athugum að þetta getur átt við um endapunkta fylkisins, ef þeir eru stærsta/minnsta gildið í fylkinu.
T.d. getur \texttt{8} hafa verið skiptistak en þá endar það aftast í fylkinu því þar bendir \texttt{j} í lokin. Þetta er jafnframt stærsta stakið í fylkinu og uppfyllir að
öll stökin í vinstra hlutfylkinu eru minna en (eða jafnt og) stakið.

Einnig eru \texttt{4} og \texttt{5} gild skiptistök, því þau uppfylla skilyrðið hér að ofan. Við sjáum aftur á móti að engin önnur stök í fylkinu gætu hafa verið skiptistak því
það má finna stak í vinstra eða hægri hlutfylki þeirra sem brýtur þetta skilyrði. Tökum þetta saman í \textsc{Töflu} 1 hér fyrir neðan.

\begin{table}[H]
\centering
\begin{tabular}{cccl}
\toprule
  \texttt{i} & \texttt{a[i]} & Vendistak? & Athugasemdir \\
  \midrule
  \texttt{0} & \texttt{2} & \xmark & \texttt{1} í hægra hlutfylki, $<$ \texttt{2} \\
  \texttt{1} & \texttt{3} & \xmark & \texttt{1} í hægra hlutfylki, $<$ \texttt{3} \\
  \texttt{2} & \texttt{1} & \xmark & sjá athugasemdir fyrir ofan \\
  \texttt{3} & \texttt{4} & \cmark & \\
  \texttt{4} & \texttt{5} & \cmark & \\
  \texttt{5} & \texttt{7} & \xmark & \texttt{6} í hægra hlutfylki $<$ \texttt{7} \\
  \texttt{6} & \texttt{6} & \xmark & \texttt{7} í vinstra hlutfylki $>$ \texttt{6} \\
  \texttt{7} & \texttt{8} & \cmark & \\
\bottomrule
\end{tabular}
\caption{Hugsanleg vendistök fyrir Quicksort}
\end{table}

\newpage

\section*{Verkefni 4}
Gefið er fylkið \texttt{[5,4,8,1,6,3,5,7]}. Sýnið hvernig það raðast með hrúguröðun (\emph{heap sort}) með því að búa til mynd
sem er sambærileg við myndina á bls. 324 í kennslubókinni (eða á glæru 35 í fyrirlestri 11). Sýnið líka hrúguna sem tvíundartré eftir að
fylkinu hefur verið hrúguraðað (þrep 1).

\newcommand*\circled[1]{\tikz[baseline=(char.base)]{
            \node[shape=circle,draw,inner sep=2pt] (char) {\texttt{#1}};}}
\newcommand*\del[1]{\tikz[baseline=(char.base)]{
            \node[shape=circle,draw=white,inner sep=2pt] (char) {\color{red}\texttt{#1}};}}
\newcommand*\circledend[1]{\tikz[baseline=(char.base)]{
            \node[shape=circle,draw=white,inner sep=2pt] (char) {\texttt{#1}};}}

\subsection*{Lausn}
\textsc{Tafla} 2 fyrir neðan sýnir keyrsluna á Heapsort og hvernig \texttt{a[]} breytist.

\newcommand*\exch[1]{\textcolor{red}#1}
\newcommand*\g{\textcolor{gray}}
\begin{table}[H]
  \centering
  \ttfamily
  \begin{tabular}{ccccccccccc}
    \multicolumn{2}{c}{} & \multicolumn{9}{c}{\texttt{a[i]}} \\
    \toprule 
    \texttt{N} & \texttt{k}  & \texttt{0} & \texttt{1} & \texttt{2} & \texttt{3} & \texttt{4} & \texttt{5} & \texttt{6} & \texttt{7} & \texttt{8} \\
    \cmidrule{3-11}
    \multicolumn{2}{c}{\scriptsize \rmfamily \itshape upphaf} &                  & \texttt{5} & \texttt{4} & \texttt{8} & \texttt{1} & \texttt{6} & \texttt{3} & \texttt{5} & \texttt{7} \\
    8          & 4           &            & 5          & 4          & 8          & \exch{7}   & 6          & 3          & 5          & \exch{1} \\
    8          & 3           &            & 5          & 4          & 8          & 7          & 6          & 3          & 5          & 1 \\
    8  & 2  & & 5        & \exch{7}   & 8          & \exch{4}   & 6 & 3 & 5 & 1 \\
    8  & 1  & & \exch{8} & 7          & \exch{5}   & 4          & 6 & 3 & 5 & 1 \\
    \multicolumn{2}{c}{\scriptsize \rmfamily \itshape röðun} & & 8        & 7          & 5          & 4          & 6 & 3 & 5 & 1 \\
    8  & 1  & & \exch{7} & \exch{6} & \g 5 & 4 & \exch{1} & \g 3 & \g 5 & \exch{8} \\
    7  & 1  & & \exch{6} & \exch{5} & \g 5 & 4 & 1 & \g 3 & \exch{7} & \color{gray} 8 \\
    6  & 1  & & \exch{5} & \exch{4} & \g 5 & \exch{3} & 1 & \g 6 & \g 7 & \g 8 \\
    5  & 1  & & \exch{5} & \g 4     & \exch{1} & \g 3 & \exch{5} & \g 6 & \g 7 & \g 8 \\
    4  & 1  & & \exch{4} & \exch{3} & \g 1 & \exch{5} & \g 5 & \g 6 & \g 7 & \g 8 \\
    3  & 1  & & \exch{3} & \exch{1} & \exch{4} & \g 5 & \g 5 & \g 6 & \g 7 & \g 8 \\
    2  & 1  & & \exch{1} & \exch{3} & \g 4 & \g 5 & \g 5 & \g 6 & \g 7 & \g 8 \\
    \multicolumn{2}{c}{\scriptsize \rmfamily \itshape raðað} & & 1 & 3 & 4 & 5 & 5 & 6 & 7 & 8 \\
    \bottomrule
  \end{tabular}
  \caption{Rakning á Heapsort}
  \label{tafla:heapsort-trace}
\end{table}

\noindent
Á næstu síðum eru myndir af breytingum á hrúgunni.

\newpage

\subsubsection*{Þrep 1 — framkalla hrúguskilyrði}
\tikzset{edge from parent/.append style={thick}}
\begin{multicols}{2}
\begin{figure}[H]
  \centering
  \Tree[ .\circled{5} [ .\circled{4} [.\circled{7} \circled{1} ] \circled{6} ] 
                      [ .\circled{8} \circled{3} \circled{5} ] ]

  \texttt{sink(4,8)}
\end{figure}

\begin{figure}[H]
  \centering
  \Tree[.\circled{5} [.\circled{4} [.\circled{7} \circled{1} ] \circled{6} ] 
                     [.\circled{8} \circled{3} \circled{5} ] ]

  \texttt{sink(3,8)}
\end{figure}

\begin{figure}[H]
  \centering
  \Tree[.\circled{5} [.\circled{7} [.\circled{4} \circled{1} ] \circled{6} ]
                     [.\circled{8} \circled{3} \circled{5} ] ]
  
  \texttt{sink(2,8)}
\end{figure}


\begin{figure}[H]
  \centering
  \Tree[.\circled{8} [.\circled{7} [.\circled{4} \circled{1} ] \circled{6} ]
                     [.\circled{5} \circled{3} \circled{5} ] ]
  
  \texttt{sink(1,8)}
\end{figure}
\end{multicols}
\noindent

\subsubsection*{Þrep 2 — raða hrúgunni}

\begin{multicols}{2}
  \begin{figure}[H]
    \centering
    \Tree[.\circled{8} [.\circled{7} [.\circled{4} \circled{1} ] \circled{6} ]
                       [.\circled{5} \circled{3} \circled{5} ] ]

    \texttt{exch(1,8), sink(1,7)}
  \end{figure}

  \begin{figure}[H]
    \centering
    \Tree[.\circled{7} [.\circled{6} [.\circled{4} \del{8} ] \circled{1} ]
                       [.\circled{5} \circled{3} \circled{5} ] ]

    \texttt{exch(1,8), sink(1,7)}
  \end{figure}

  \begin{figure}[H]
    \centering
    \Tree[.\circled{6} [.\circled{5} [.\circled{4} \del{8} ] \circled{1} ]
                       [.\circled{5} \circled{3} \del{7} ] ]

    \texttt{exch(1,7), sink(1,6)}
  \end{figure}

  \begin{figure}[H]
    \centering
    \Tree[.\circled{5} [.\circled{4} [.\circled{3} \del{8} ] \circled{1} ]
                       [.\circled{5} \del{6} \del{7} ] ]

    \texttt{exch(1,6), sink(1,5)}
  \end{figure}

  \begin{figure}[H]
    \centering
    \Tree[.\circled{5} [.\circled{4} [.\circled{3} \del{8} ] \del{5} ] 
                       [.\circled{1} \del{6} \del{7} ] ]

    \texttt{exch(1,5), sink(1,4)}
  \end{figure}

  \begin{figure}[H]
    \centering
    \Tree[.\circled{4} [.\circled{3} [.\del{5} \del{8} ] \del{5} ] 
                       [.\circled{1} \del{6} \del{7} ] ]

    \texttt{exch(1,4), sink(1,3)}
  \end{figure}

  \begin{figure}[H]
    \centering
    \Tree[.\circled{3} [.\circled{1} [.\del{5} \del{8} ] \del{5} ] 
                       [.\del{4} \del{6} \del{7} ] ]

    \texttt{exch(1,3), sink(1,2)}
  \end{figure}

  \begin{figure}[H]
    \centering
    \Tree[.\circled{1} [.\del{3} [.\del{5} \del{8} ] \del{5} ] 
                       [.\del{4} \del{6} \del{7} ] ]

    \texttt{exch(1,2), sink(1,1)}
  \end{figure}

\end{multicols}

\begin{figure}[H]
  \centering
  \Tree[.\circledend{1} [.\circledend{3} [.\circledend{5} \circledend{8} ] \circledend{5} ] 
                        [.\circledend{4} \circledend{6} \circledend{7} ] ]
  \caption{Fullröðuð hrúga eftir Heapsort}
\end{figure}

\end{document}
