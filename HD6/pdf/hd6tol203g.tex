\documentclass[12pt, a4paper, hidelinks]{article}

\usepackage[icelandic]{babel}
\usepackage[T1]{fontenc}
\usepackage[utf8]{inputenc}


\usepackage{amsmath, amssymb, amsfonts}
\usepackage{mathtools}

\usepackage{minted}
\usemintedstyle{default}
\renewcommand{\listingscaption}{Forrit}

\usepackage{url}
\usepackage{hyperref}
\usepackage[hang, flushmargin]{footmisc}
\usepackage[labelfont=sc]{caption}


\usepackage[svgnames]{xcolor}
\usepackage{tabularx}
\usepackage{float}
\usepackage{graphicx}
\usepackage{booktabs}
\usepackage{enumerate}
\usepackage{multirow}
\usepackage{tikz}
\usepackage{tikz-qtree}
\usepackage{pifont}
\usepackage{multicol}
\usepackage{tcolorbox}

\newcommand{\cmark}{\color{Green}\ding{51}}
\newcommand{\xmark}{\color{Red}\ding{55}}

\usepackage{times, mathptmx}
\usepackage[scaled=0.85]{beramono}

\usepackage{fancyhdr}
\pagestyle{fancy}
\fancyhf{}
\fancyhead[L]{Kári Hlynsson}
\fancyhead[C]{TÖL203G HEIMADÆMI 5}
\fancyhead[R]{\today}
\fancyfoot[C]{\thepage}

\newcommand{\doctitle}{\uppercase{Heimadæmi 5}}
\newcommand{\coursename}{Tölvunarfræði 2}
\newcommand{\coursenum}{TÖL203G}

% ——— Mengjatákn
\newcommand{\N}{\mathbb{N}}
\newcommand{\Z}{\mathbb{Z}}
\newcommand{\Q}{\mathbb{Q}}
\newcommand{\R}{\mathbb{R}}
\newcommand{\C}{\mathbb{C}}

% ——— Vigrar
\renewcommand{\u}{\mathbf{u}}
\renewcommand{\v}{\mathbf{v}}
\renewcommand{\b}{\mathbf{b}}
\newcommand{\w}{\mathbf{w}}
\newcommand{\p}{\mathbf{p}}
\newcommand{\x}{\mathbf{x}}
\newcommand{\y}{\mathbf{y}}
\newcommand{\z}{\mathbf{z}}

\begin{document}
\thispagestyle{plain}
\centerline{\bfseries\Large\doctitle}
\medskip
\centerline{\large\coursenum\ \coursename}
\bigskip

\centerline{\large Kári Hlynsson\footnote{Slóð á Github kóða: \url{https://github.com/lvthnn/TOL203G/tree/master/HD5}}}
\bigskip
\centerline{Háskóli Íslands}
\medskip
\centerline{\today}

\section*{Verkefni 1}
Þið eigið að breyta táknatöfluútfærslunni \textsc{SequentialSearchST.java}, þannig að
listinn sé sjálfskipandi (\emph{self-organizing}). Sjálfskipandi gagnagrindur laga sig
að notkunarmynstri notandans, þannig að lyklar sem oft er leitað að finnast hraðar en
þeir sem sjaldan er leitað að. Þið eigið að útfæra tiltekna útgáfu sem kallast færa-fremst
(\emph{move-to-front}). Hún felst í því að þegar kallað er á \texttt{get(k)}, þá er hnúturinn
með lyklinum \texttt{k} færður fremst í tengda listann (ef hann finnst). Það þýðir að ef leitað
er aftur að \texttt{k} fljótlega þá finnst hann hratt.

Þið eigið að skila breytta fallinu \text{get} og skjáskoti af keyrslu á \texttt{main}-fallinu fyrir
inntakið \texttt{A B R A C A D A B R A}, sem þið sláið inn eða pípið úr skrá. Útkoman ætti að vera
\texttt{D 6, C 4, R 9, B 8, A 10}, þ.e. sætisnúmerin á síðasta tilvikinu af hverjum staf.
\end{document}
