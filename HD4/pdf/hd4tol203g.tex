\documentclass[12pt, a4paper, hidelinks]{article}

\usepackage[icelandic]{babel}
\usepackage[T1]{fontenc}
\usepackage[utf8]{inputenc}


\usepackage{amsmath, amssymb, amsfonts}
\usepackage{mathtools}

\usepackage{minted}
\renewcommand{\listingscaption}{Forrit}

\usepackage{url}
\usepackage{hyperref}
\usepackage[hang, flushmargin]{footmisc}

\usepackage{xcolor}
\usepackage{tabularx}
\usepackage{graphicx}
\usepackage{booktabs}
\usepackage{enumerate}
\usepackage{tikz}

\usepackage{times, mathptmx}
\usepackage[scaled=0.85]{beramono}

\usepackage{fancyhdr}
\pagestyle{fancy}
\fancyhf{}
\fancyhead[L]{Kári Hlynsson}
\fancyhead[C]{TÖL203G HEIMADÆMI 4}
\fancyhead[R]{\today}
\fancyfoot[C]{\thepage}

\newcommand{\doctitle}{\uppercase{Heimadæmi 3}}
\newcommand{\coursename}{Tölvunarfræði 2}
\newcommand{\coursenum}{TÖL203G}

% ——— Mengjatákn
\newcommand{\N}{\mathbb{N}}
\newcommand{\Z}{\mathbb{Z}}
\newcommand{\Q}{\mathbb{Q}}
\newcommand{\R}{\mathbb{R}}
\newcommand{\C}{\mathbb{C}}

% ——— Vigrar
\renewcommand{\u}{\mathbf{u}}
\renewcommand{\v}{\mathbf{v}}
\renewcommand{\b}{\mathbf{b}}
\newcommand{\w}{\mathbf{w}}
\newcommand{\p}{\mathbf{p}}
\newcommand{\x}{\mathbf{x}}
\newcommand{\y}{\mathbf{y}}
\newcommand{\z}{\mathbf{z}}

\title{}

\begin{document}
\thispagestyle{plain}
\centerline{\bfseries\Large\doctitle}
\medskip
\centerline{\large\coursenum\ \coursename}
\bigskip

\centerline{\large Kári Hlynsson\footnote{Slóð á Github kóða: \url{https://github.com/lvthnn/TOL203G/tree/master/HD4}}}
\bigskip
\centerline{Háskóli Íslands}
\medskip
\centerline{\today}

\bigskip

\section*{Verkefni 1}
Þetta dæmi byggir á æfingadæminu úr dæmatímunum.

\begin{enumerate}[(a)]
    \item Bætið við klasann \texttt{Card} úr æfingadæminu aðferðinni \texttt{toString()}, sem skilar
    streng með gildi spilsins sem hægt er að prenta út. Þið getið notað ensku upphafsstafina fyrir spaða
    (S), hjarta (H), tígul (D) og lauf (C). Sömuleiðis fyrir mannspilin: ás (A), kóngur (K), drottning (Q)
    og gosi (J). Þannig á aðferðin að skila ,,H-K'' fyrir hjartakóng, ,,C-5'' fyrir laufafimmu o.s.frv.

    \item Skrifið forritið \texttt{CardDeal}, sem tekur á skipanalínunni töluna $k$ sem er á bilinu 1 til 52.
    Forritið prentar þá út $k$ spil sem valin eru af handahófi úr spilastokki. Til þess að við prentum ekki sömu
    spilin út aftur, þá er best að búa til 52-spila fylki. Það er fyllt af öllum mögulegum spilum í venjulegri röð
    og þetta fylki er síðan stokkað. Til þess getið þið notað aðferðina shuffle úr \texttt{StdRandom}. Síðan prentar
    forritið út $k$ fyrstu stökin í fylkinu. Skilið kóðanum fyrir forritið og skjáskoti af keyrslu.
\end{enumerate}

\subsection*{Lausn}
Hér kemur lausn.

\newpage

\section*{Verkefni 2}
Í útfærslunni á Valröðun í bókinni þá er ekki athugað hvort við þurfum að víxla á stökum \texttt{a[i]} og \texttt{a[min]}.
Ef \texttt{i} er jafnt og \texttt{min} þá er þessi víxlun óþörf. Bætið við \texttt{if}-setningu á undan víxlunarskipuninni
í röðunarfallinu \texttt{sort} sem athugar hvort það þurfi að víxla. Bætið tímamælingarkóða við báðar útgáfurnar og keyrið þær
svo á skránni \texttt{32Kints.txt} og athugið hvort það sé einhver hraðamunur. Þið ættuð að keyra hvora útgáfu a.m.k. þrisvar
sinnum og taka meðaltalið, því það er alltaf einhver breytileiki í keyrslutímanum. Skilið breytta fallinu og niðurstöðum
tímamælinganna.

% meðaltal sort original 2.1794999999999995
% meðaltal sort fast	    0.6352

\newpage

\section*{Verkefni 3}
Þetta er spurning um hegðun valröðunar og innsetningarröðunar á tilteknu $N$-staka inntaksfylki:
\begin{enumerate}[(a)]
    \item Öll stökin í fylkinu hafa sama gildi:
    \begin{enumerate}[(i)]
        \item Hversu marga samanburði notar valröðun?
        \item Hversu marga samanburði notar innsetningarröðun?
    \end{enumerate}
    \item Fylkið er óraðað en inniheldur aðeins tvö ólík gildi, $A$ og $B$. Fjöldi staka af hvoru
    gildi er óþekktur:
    \begin{enumerate}[(i)]
        \item Hversu marga samanburði notar valröðun?
        \item Hversu marga samanburði notar innsetningarröðun?
    \end{enumerate}
\end{enumerate}

\subsection*{Lausn}
\subsubsection*{Hluti (a)}
    \begin{enumerate}[(i)]
        \item Valröðun notar alltaf sama fjölda samanburða því það ítrar frá vísinum \texttt{i}
        og út í enda fylkisins. Fjöldi samanburða er því sem áður $(N - 1) + (N - 2) + \cdots + 2 + 1 \sim N^2/2$.
        \item Innsetningarröðun stöðvar alltaf ef \texttt{a[j]} $\leq$ \texttt{a[j - 1]}, svo í þessu tilviki myndum við alltaf
        bara líta eitt stak aftur fyrir okkur í röðun á fylkinu. Því er heildarfjöldi samanburða $N - 1 \sim N$.
    \end{enumerate}
\subsubsection*{Hluti (b)}
    \begin{enumerate}[(i)]
       \item Eins og áður sagði er valröðun ekki háð inntaki og því höfum við aftur fjölda samanburða $(N - 1) + (N - 2) + \cdots + 2 + 1 \sim N^2/2$.
       \item Látum $n$ og $m$ tákna fjöldann af gildunum $A$ og $B$ í fylkinu, í þeirri röð. Þá er $N = n + m$ heildarföldi staka í fylkinu. Án skerðingar á
       víðgildni athugum við að ef $n \gg m$ þá erum við nokkurn veginn að fást við (ii) í (a)-lið. Ef svo er ekki og $a \approx m$ þá er fjöldi samanburða áður
       en stak finnur sinn stað í raðaða fylkinu $\frac 12 (N - i)$. Við fáum því að heildarfjöldi samanburða er $\frac 12 (N - 1) + \frac 12 (N - 2) + \cdots + 1 + \frac 12 = N^2/4$.
    \end{enumerate}

\newpage

\section*{Verkefni 4}
Við getum skilgreint röðunarreikniritið \emph{Slembiröðun}, sem virkar þannig að á meðan fylkið
er ekki raðað þá veljum við tvo vísa \texttt{i} og \texttt{h} af handahófi (á milli $0$ og $N - 1$). Ef
stök \texttt{a[i]} og \texttt{a[j]} eru í langri röð í fylkinu þá víxlum við á þeim og höldum áfram. Forritið
þetta reiknriti í Java (þið getið notað \texttt{Selection.java} sem fyrirmynd). Takið tímann á keyrslu á \texttt{1Kints.txt}.
Keyrið forritið ykkar a.m.k. 5 sinnum og skoðið breytileikann á tímanum. Skilið Java kóðanum fyrir fallið og tímunum á keyrslunum.

\newpage

\section*{Verkefni 5}
Nota á Shell röðun með $3x + 1$ skrefstærðum á 10-staka fylki. Þá eru tvær skrefstærðir: 1 og 4 (reyndar er byrjað með $h = 4$).
\begin{enumerate}[(a)]
    \item Hvert er besta inntak fyrir þessa tegund af Shellröðun? Rökstyðjið og sýnið heildarfjölda
    samanburða fyrir 10-staka fylki.
    \item Sýnið hvernig þessi Shell röðun virkar á 10-staka fylki í öfugri röð (t.d. 10, 9, $\ldots$\@ , 2, 1).
    Sýnið fylkið eftir hvora umferð og fjölda samanburða. Berið fjölda samanburða hér saman við fjölda samanburða
    sem innsetningarröðun myndi nota á þessu fylki.
\end{enumerate}

\end{document}